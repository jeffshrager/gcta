\documentstyle[12pt,doublespace,bezier]{article}
\setlength{\oddsidemargin}{0in}
\setlength{\textwidth}{6.5in}
\setlength{\topmargin}{-0.5in}
\setlength{\textheight}{9in}
\setstretch{1.3}
\begin{document}

\vspace*{2cm}
  \begin{center}
    {\Large \bf Programs for Computing Group Sequential Boundaries Using
    the Lan-DeMets Method} \\[2ex]
        {\large Version 2} \\[2ex]
           {\bf by} \\[2ex]

      \begin{tabular}{c}
        {\large \bf David M. Reboussin} \\
        Department of Public Health Sciences \\
        Bowman Gray School of Medicine       \\
        Winston-Salem, NC 27157             \\
      \end{tabular}

\vspace{.5cm}
      \begin{tabular}{c}
        {\large \bf David L. DeMets}    \\
        Departments of Statistics and Biostatistics      \\
        University of Wisconsin          \\
        Madison, WI 53706                \\
      \end{tabular}

\vspace{.5cm}
      \begin{tabular}{c}
        {\large \bf KyungMann Kim}       \\
        Department of Biostatistics      \\
        Harvard School of Public Health
        and Dana-Farber Cancer Institute   \\
        Boston, MA 02115              \\
      \end{tabular}

\vspace{.5cm}
      \begin{tabular}{c}
        {\large \bf K. K. Gordon Lan}  \\
        Department of Statistics/Computer and Information Systems \\
        George Washington University \\
        Washington, DC 20052          \\
      \end{tabular}

\vspace{.5cm}
\today
\end{center}


\newpage
\begin{figure}
\begin{abstract}
FORTRAN programs for the computation of boundaries and exit probabilities
in the sequential analysis of clinical trials are described.  The
computations are appropriate for any trial based on normally distributed
test statistics with independent increments, including those in which
patients give a single continuous or binary response, survival studies, and
certain longitudinal designs.  Interim analyses need not be equally spaced,
and their number need not be specified in advance.  In addition to
boundaries, power computations, probabilities associated with a given set
of boundaries, and confidence intervals can be computed.  An explanation of
the input and output, along with some verbatim transcripts of interactive
sessions are included.  Some theoretical background and description of the
numerical computations is provided in the appendix.  The program and this
documentation are updates of earlier versions.

\vspace{.5cm}
  {\bf KEY WORDS}:
  Group sequential testing;
  Statistical computing;
  Type I error spending function;
  Use function.
\end{abstract}
\end{figure}


%\clearpage
\section{Introduction}

The design of many clinical trials includes some strategy for early
stopping if an interim analysis reveals large differences between treatment
groups.  In addition to saving time and resources, such a design feature
can reduce study participants' exposure to the inferior treatment.
However, when repeated significance testing on accumulating data is done,
some adjustment of the usual hypothesis testing procedure must be made to
maintain an overall significance level (Armitage, McPherson \& Rowe, 1969;
McPherson \& Armitage, 1971).  The methods described by Pocock (1977) and
O'Brien \& Fleming (1979), among others, are popular implementations of
group sequential testing for clinical trials.  Sometimes interim analyses
are equally spaced in terms of calendar time or the information available
from the data, but this assumption can be relaxed to allow for unplanned or
unequally spaced analyses.  Lan \& DeMets (1983) introduced type I error
spending functions, denoted $\alpha^{\star}$, and determined boundaries by
\begin{eqnarray}
  {\rm Pr}
  (Z_1 \geq b_1 \, {\rm or} \,
         \cdots     \, {\rm or} \,
   Z_k \geq b_k)
  =
  \alpha^{\star}(\tau)
  \label{eqn:ldm}
\end{eqnarray}
where
\begin{math}
  b_1,\,\ldots,\,b_k,
\end{math}
are (upper) boundaries for the sequence of interim test statistics and
$\tau$ is either the proportion of elapsed time to maximum duration or
observed information to total information.  That is, if the interim
standardized test statistic at the $k^{th}$ interim analysis is denoted by
$Z_k$, we continue the trial as long as $| Z_k | < b_k$ (two-sided),
otherwise termination is considered.  The spending function
$\alpha^{\star}(\tau)=0$ for $\tau=0$ and $\alpha^{\star}(\tau) = \alpha$
for $\tau = 1$.  That is, this flexible procedure guarantees a fixed
$\alpha$ level when the trial is complete.  Neither the time or the number
of analyses needs to be specified in advance: only $\alpha^{\star}(\tau)$
must be specified.  Issues surrounding the use of calendar time and
information have been discussed by Lan \& DeMets (1989) and Lan, Reboussin
\& DeMets (1994).  Spending functions, which are also called use functions,
are prespecified and correspond to those described by Lan \& DeMets (1983)
and Kim \& DeMets (1987a).  These are similar to commonly used group
sequential boundaries proposed by Pocock (1977) and O'Brien \& Fleming
(1979).  Additional spending functions may be found in Hwang, Shih \& de
Cani (1990).

\begin{figure}
\begin{center}
\setlength{\unitlength}{0.8in}
\begin{picture}(5,4.0)(0.0,0.0)

% Vertical axis

 \put (0.00,0.00){\line(0,1){4.0}}
 \put (-.0625,0.00){\rule{8pt}{.5pt}}
 \put (-.0625,0.50){\rule{8pt}{.5pt}}
 \put (-.0625,1.00){\rule{8pt}{.5pt}}
 \put (-.0625,1.50){\rule{8pt}{.5pt}}
 \put (-.0625,2.00){\rule{8pt}{.5pt}}
 \put (-.0625,2.50){\rule{8pt}{.5pt}}
 \put (-.0625,3.00){\rule{8pt}{.5pt}}
 \put (-.0625,3.50){\rule{8pt}{.5pt}}
 \put (-.0625,4.00){\rule{8pt}{.5pt}}
 \put (-.5000,3.75){{\normalsize Z}}

 \put (-.35,0.00){-4.0}
 \put (-.35,0.50){-3.0}
 \put (-.35,1.00){-2.0}
 \put (-.35,1.50){-1.0}
 \put (-.35,2.00){ 0.0}
 \put (-.35,2.50){ 1.0}
 \put (-.35,3.00){ 2.0}
 \put (-.35,3.50){ 3.0}
 \put (-.35,4.00){ 4.0}

% Horizontal axis

 \put (0.00,2.00){\vector(1,0){5}}
 \put (0.50,2.00){\rule[-4pt]{.5pt}{8pt}}
 \put (1.30,2.00){\rule[-4pt]{.5pt}{8pt}}
 \put (1.45,2.00){\rule[-4pt]{.5pt}{8pt}}
 \put (2.00,2.00){\rule[-4pt]{.5pt}{8pt}}
 \put (3.00,2.00){\rule[-4pt]{.5pt}{8pt}}
 \put (5.25,2.25){{\normalsize elapsed}}
 \put (4.95,2.10){{\normalsize time or information}}

% Dots

 \put (0.50,2.20){\circle*{.05}}
 \put (1.30,3.10){\circle*{.05}}
 \put (1.45,2.80){\circle*{.05}}
 \put (2.00,2.90){\circle*{.05}}
 \put (3.00,3.20){\circle*{.05}}

% Boundaries

 \put (0.00,0.25){\line(1,0){1}}
 \put (0.00,3.75){\line(1,0){1}}
 \put (4.50,1.00){\line(0,1){2}}

 \bezier{200}(1.00,3.75)(2.50,3.00)(4.50,3.00)
 \bezier{200}(1.00,0.25)(2.50,1.00)(4.50,1.00)
\end{picture}
\end{center}
\label{fig:boundaries}
\caption{Sequential outcomes and boundaries for interim standardized
test statistics from a clinical trial.}
\end{figure}

\subsection*{Options}

The program described here perform computations related to group sequential
boundaries, such as the one illustrated in Figure~\ref{fig:boundaries}.  The
program begins by prompting the user to specify whether it is being run
interactively or not, and then to specify one of four options.  It
continues prompting based on the selected option.  The options are:
\begin{itemize}
\item computation of boundaries for a specified spending function
      (including graphical presentation);
\item power calculation for a specified set of boundary values and a
      drift parameter corresponding to the alternative hypothesis;
\item computation of the exit probabilities for a specified spending
      function, analysis times, and drift parameter;
\item computation of confidence intervals following termination of a trial.
\end{itemize}
For interim analysis of an ongoing clinical trial, the first option takes
as input the times of the previous and current interim analyses, and the
type I error spending function.  The program then reports what boundaries
should be used to determine whether or not to stop the trial.  This is
accomplished using Equation (\ref{eqn:ldm}) and a searching routine which
makes an initial choice of boundaries, computes stopping probabilities, and
alters boundaries until the desired alpha level is obtained.  The other
options, in contrast, evaluate probabilities associated with a given set of
boundaries.  They require as input boundaries and times for the interim
analyses.
The package can be used to design sequential trials, determine boundary
values while the trial is ongoing or compute confidence intervals when the
trial is ended.  We present examples for design or analyses using test
statistics comparing mean, binomial, survival or repeated measures
outcomes.

\subsection*{Summary of methodology}

A detailed presentation of the methodology may be found in Lan \& DeMets
(1983), DeMets \& Lan (1984), and Lan \& Zucker (1993).  Group sequential
procedures for interim analyses are equivalent to discrete boundary
crossing problems for a Brownian motion process $W(t)$ with drift parameter
$\theta$.  We take advantage of this correspondence in both theoretical
developments and in implementation.  At each interim analysis, a
standardized test statistic $Z_k$ is computed.  These normally distributed
variates $Z_1, \ldots, Z_k$ have mean $\theta \sqrt{\tau_k}$, where $\theta$
is the ``drift'' parameter, and for $j \leq k$,
\begin{math}
  {\rm Cov}(Z_j,Z_k) = \sqrt{\tau_j / \tau_k}
\end{math}
where $\tau_k$ is the information fraction (or information time) at the
$k^{th}$ analysis, e.g.\ $\tau_k = n_k/n_K$ if $n_K$ is the maximum sample
size (per arm).  The drift parameter $\theta$ and the standardized
difference $\delta_S$ are related by the equation
\begin{math}
  {\rm E}(Z_K) = \theta = \delta_S \sqrt{n_K}.
\end{math}

To reiterate in more technical terms, the program uses Equation
(\ref{eqn:ldm}) to determine one of
\begin{itemize}
  \item $b_1,\,\ldots,\,b_k$
        for given $\alpha^{\star}(\tau)$, and $\tau_1,\,\ldots,\,\tau_k$, 

  \item $\theta$ given
        $p_k$, $b_1,\,\ldots,\,b_k$ and $\tau_1,\,\ldots,\,\tau_k$

  \item $p_1, \ldots, p_K$, where
        $p_k = \Pr(Z_1 \geq b_1 \, {\rm or} \, \ldots \, {\rm or} \, Z_k
        \geq b_k)$, 
        given $b_1,\,\ldots,\,b_k$ and $\tau_1, \ldots, \tau_k$,

  \item a confidence interval for $\theta$ given
        $b_1,\,\ldots,\,b_k$, $\tau_1,\,\ldots,\,\tau_k$,
        and $Z_k$
\end{itemize}


It may be useful to note correspondences between the notation used here and
in some other references (see Table 1).
\begin{table}
\caption{Correspondence of notation for commonly used group sequential
parameters.}
\begin{center}
\begin{tabular}{c|c|c|c} \hline
  \multicolumn{1}{c}{Parameter}      &
  \multicolumn{1}{c}{}               &
  \multicolumn{1}{c}{Kim and DeMets} &
  \multicolumn{1}{c}{Pocock}         \\ \hline
  Standardized difference  & $\delta_S$ & $\zeta$ & $\delta / \sqrt{2} \sigma$\\
  Drift of Brownian motion & $\theta$   & $\xi$   & $\Delta \sqrt{N}$         \\
  Pocock noncentrality & $\theta / \sqrt{K}$ & $\xi / \sqrt{K}$ & $\Delta$ \\
  Accumulated sample size per arm & $n_k$& $n_k$   & $ni$               \\
  Maximum sample size per arm & $n_K$ or $nK$ & $n_K$ & $nN$ \\
\hline
\end{tabular}
\end{center}
\end{table}
To clarify notation for the sample size, let $n_k$ be the number of
subjects at the $k^{th}$ look in each treatment arm.  $n_K$ is the maximum
number of subjects per treatment arm and $K$ is the maximum number of looks
or interim analyses.  If there are $n$ subjects accumulated between interim
analyses, $n_K = nK$.  The drift parameter $\theta$ can be expressed in
terms of the noncentrality parameter $\Delta$ in Pocock (1977) as $\theta =
\Delta \sqrt{K}$.



\section{Use of the Program for Study Design}

Although spending functions provide flexibility in data monitoring and
do not require analysis times to be prespecified, the anticipated number
and timing of interim analyses must be specified for design purposes.
This is not more restrictive than for the group sequential
procedures proposed by Pocock (1977) or O'Brien \& Fleming (1979).
Deviation from the initial design, even substantially, does not cause
a serious loss of power.  Thus for design only, we shall assume
$n_k = nK$, where $K$ is the anticipated number of interim analyses and $n$
is the anticipated number of subjects accrued between analyses.

Kim \& DeMets (1992) provide a detailed discussion of sample size
determination for group sequential testing.  The relationship between
sample size and power depends on two quantities: the drift parameter of the
underlying Brownian motion and the standardized difference between control
and treatment arms.  Thus by determining $\theta$ and $\delta_S$ for a
particular design problem, the required sample size can be computed.  The
value of $\theta$ depends on the desired power, the set of boundaries and
analysis times, and the properties of Brownian motion.  Exit or rejection
probabilities for Brownian motion given a set of boundaries can be computed
by the program or, for certain designs, found in the tables provided by Kim
\& DeMets (1992).  The sequential boundaries are determined by the choice
of spending function $\alpha^{\star}(\tau)$, the number and timing of
interim analyses, the $\alpha$ level and whether the test is one or two
sided.  The standardized difference $\delta_S$, on the other hand, depends
on the type of data to be collected by the study.  Several examples are
detailed below for normal, binomial and survival data.

Kim \& DeMets (1992) provide tables of drift parameters for spending
functions producing O'Brien-Fleming type and Pocock type boundaries
($\alpha_1^{\star}(\tau)$ and $\alpha_2^{\star}(\tau)$, respectively).  The
program currently offers five choices for spending functions, but others
can be added (see Appendix).


\subsection{Normally distributed data}

\subsubsection{Kim-DeMets example}

Kim \& DeMets (1992) discuss the following example.  Suppose that a
normally distributed response has mean in controls of $\mu_C = 220$ with
standard deviation $\sigma = 30$.  The null hypothesis is $H_0: \mu_C
= \mu_E$, where $\mu_E$ is the mean in the experimental group,
expected to be 200.  The test statistic is
\begin{displaymath}
  Z_k = \frac{\bar x_C - \bar x_E}{\sqrt{2 \sigma^2/n_k}}.
\end{displaymath}
Then the drift parameter is
\begin{displaymath}
  \theta =
  \frac{\sqrt{n_K}(\mu_C - \mu_E)}{\sqrt{2\sigma^2}} =
  \sqrt{n_K} \, \delta_S.
\end{displaymath}
So
\begin{displaymath}
  n_K =
  \frac{2 \sigma^2}{(\mu_C - \mu_E)^2} \times \theta^2.
\end{displaymath}
For the program, we specify two-sided $\alpha= 0.05$ O'Brien-Fleming type
($\alpha_1^{\star}$) boundaries with $K = 5$ looks at 0.2, 0.4, 0.6, 0.8
and 1.0 (see Section 4.1).  The output boundary values are
\begin{center}
\begin{tabular}{cc} \hline
\multicolumn{1}{c}{$\tau$}&
\multicolumn{1}{c}{O'Brien-Fleming type boundary values} \\ \hline
 0.2 & $\pm$ 4.8769  \\
 0.4 & $\pm$ 3.3569  \\
 0.6 & $\pm$ 2.6803  \\
 0.8 & $\pm$ 2.2898  \\
 1.0 & $\pm$ 2.0310  \\
\hline
\end{tabular}
\end{center}
Kim \& DeMets (1992) indicate that $\theta = 3.28$ for 90\% power, so
\begin{displaymath}
  n_K = \frac{2(30)^2}{(220-200)^2} \times (3.28)^2 = 48.41.
\end{displaymath}
The program can verify that $\theta = 3.28$ corresponds to 90\% power, and
that alternative timings of analyses does not greatly affect the power (see
Section 4.1).  The effect of alternative assumptions for $\mu_E$ on sample
size can be determined without recomputing $\theta$.


\subsubsection{Kim-DeMets example with Pocock boundary}

Suppose that in the previous example the O'Brien-Fleming type boundaries were
replaced with Pocock type boundaries ($\alpha_2^{\star}$).  The computations
are identical except for the value of $\theta$.  Two-sided 0.05 Pocock
type boundary values are
\begin{center}
\begin{tabular}{cc} \hline
\multicolumn{1}{c}{$\tau$}&
\multicolumn{1}{c}{Pocock type boundary values} \\ \hline
 0.2 & $\pm$ 2.4380  \\
 0.4 & $\pm$ 2.4268  \\
 0.6 & $\pm$ 2.4101  \\
 0.8 & $\pm$ 2.3966  \\
 1.0 & $\pm$ 2.3859  \\
\hline
\end{tabular}
\end{center}
Kim and DeMets (1992) indicate that for 90\% power using these boundaries,
$\theta = 3.55$, so
\begin{displaymath}
  n_K = \frac{2(30)^2}{(220-200)^2} \times (3.55)^2 = 56.71.
\end{displaymath}


\subsubsection{An example using Pocock's notation}

We duplicate an example from Pocock (1977).  If we take $\alpha = 0.05$ and
$N = 5$, corresponding boundaries are determined.  For a desired power
$1-\beta = 0.90$, we determine using the program that $\Delta \sqrt{N} = 3.28$
so that $\Delta = 1.59$.  To compare two sample means, we compute
\begin{displaymath}
  Z_i^* = (\bar x_E - \bar x_C) \sqrt{n i} / \sqrt{2 \sigma^2}
\end{displaymath}
where $i=1,\ldots,N$ and $H_0: \mu_C = \mu_E$ and from Pocock (1977)
\begin{displaymath}
  \Delta = \sqrt{n} (\mu_E - \mu_C) / \sqrt{2 \sigma^2}.
\end{displaymath}
For $H_A: \mu_E - \mu_C = 0.5 \sigma$,
\begin{displaymath}
  n = \frac{2 \sigma^2}{(\mu_E - \mu_C)^2} \times \Delta^2
    = \frac{2 \sigma^2}{(0.5 \sigma)^2} \times  (1.59)^2
    = 20.22,
\end{displaymath}
so $2nN = 2(20)(5) = 200$ subjects.


\subsection{Binomially distributed data}

In the binomial case, where we test $H_0: p_E = p_C$, assume $p_C =
0.4$ and $p_E = 0.6$.  The statistic
\begin{displaymath}
  Z_k = (\hat p_E - \hat p_C) /
        \sqrt{\hat p_E (1-\hat p_E) + \hat p_C (1 - \hat p_C)}
\end{displaymath}
has asymptotically a normal distribution with a mean of 0 and a variance
of 1 (under $H_0: p_C = p_T$).  The standardized difference is
\begin{displaymath}
  \delta_S = (p_E - p_C) / \sqrt{p_E (1-p_E) + p_C (1 - p_C)}
     \approx (p_E - p_C) / \sqrt{2 \bar p (1- \bar p)}
\end{displaymath}
where $\bar p = (p_E + p_C)/2.$

\subsubsection{Kim and DeMets example}

Kim \& DeMets (1992) show
\begin{math}
  \theta = {\rm E} (Z_K)
  = \sqrt{n_K} \delta_S
  = {\sqrt{n_K} (p_C - p_E)}/{\sqrt{2 \bar p (1- \bar p)}}
\end{math}
so
\begin{displaymath}
  n_K = \frac{2 \bar p (1- \bar p)}{(p_C - p_E)^2} \times \theta^2.
\end{displaymath}
For example, if $p_C=0.4$ and $p_E=0.6$ under the alternative hypothesis,
then $\bar p = 0.5$, and for a one sided $\alpha=0.05$ test using five
interim analyses and Pocock type boundaries ($\alpha_2^{\star}$), we have
\begin{center}
\begin{tabular}{cc} \hline
\multicolumn{1}{c}{$\tau$}&
\multicolumn{1}{c}{Pocock type Z values} \\ \hline
 0.2 &  +2.1762  \\
 0.4 &  +2.1437  \\
 0.6 &  +2.1132  \\
 0.8 &  +2.0895  \\
 1.0 &  +2.0709  \\
\hline
\end{tabular}
\end{center}
For $H_A: p_E > p_C$ and 90\% power, Kim \& DeMets (1992) report
$\theta = 3.21$ (or see Section 4.2), so 
\begin{displaymath}
  n_K = \frac{2 (.5)(.5)}{(.2)^2} \times (3.21)^2= 128.80.
\end{displaymath}


\subsubsection{O'Brien-Fleming example}

As another binomial example, consider a two sided $\alpha = 0.05$ test with
O'Brien-Fleming type ($\alpha_1^{\star}$) boundaries, and for design
purposes only, assume $K=5$ equally spaced analyses at 0.2, 0.4, 0.6, 0.8
and 1.0.  As above, we take $H_0: p_C = p_E$, but now let $p_C = 0.11$ and
$p_E = 0.0825$ under the alternative hypothesis (a 25\% reduction,
$\bar p = 0.096$).  The program produces
\begin{center}
\begin{tabular}{cc} \hline
\multicolumn{1}{c}{$\tau$}&
\multicolumn{1}{c}{O'Brien-Fleming type Z values} \\ \hline
 0.2 & $\pm$ 4.8769  \\
 0.4 & $\pm$ 3.3569  \\
 0.6 & $\pm$ 2.6803  \\
 0.8 & $\pm$ 2.2898  \\
 1.0 & $\pm$ 2.0310  \\
\hline
\end{tabular}
\end{center}
From Kim and DeMets (1992), $\theta = 3.28$ (see Section 4.1) so
\begin{displaymath}
  n_K = \frac{2  (0.096) (0.904)}{(0.028)^2} \times (3.28)^2 = 2381.78.
\end{displaymath}


\subsection{Survival data}

Suppose we are interested in comparing the hazard rate of two populations.
Let $\lambda_0(u)$ be the hazard function of the control group and
$\lambda_1(u)$ the hazard function in the treatment group.  Under the
null hypothesis $\lambda_0 = \lambda_1$ and $\phi = \log (\lambda_1
/\lambda_0) = 0$.  The logrank statistic is
\begin{displaymath}
  L(d)= \sum_{i=1}^d \left( x_i \frac{r_{ic}}{r_{ic}+r_{it}}   \right)
\end{displaymath}
where $d$ is the number of events, $x_i$ is 1 if the event at $t_i$ is in
the control group and 0 if it is in the treatment group, $r_{ic}$ is the
number of patients in the control group at risk just before $t_i$, and
$r_{it}$ is the number of patients in the treatment group at risk just
before $t_i$.  The expected value of $L(d)$ is approximately
$\phi \times (d/4)$, and the estimated variance is
\begin{displaymath}
   \sum_{i=1}^d \left(   \frac{r_{ic}}{r_{ic}+r_{it}} \right)
                \left(1- \frac{r_{ic}}{r_{ic}+r_{it}} \right)
   \approx d/4
\end{displaymath}
These approximations are reasonable if $r_{it} \approx r_{ic}$ and $\phi$
is close to 0.  If $d_k$ is the number of events at analysis $k$, the
statistic
\begin{math}
  Z_k = L(d_k)/\sqrt{d_k/4}
\end{math}
has a $N(\phi \sqrt{d_k/4}, 1)$ distribution, so
\begin{math}
  \theta = \phi \sqrt{d_K/4}.
\end{math}
Then the maximum number of events required per arm is
\begin{displaymath}
  d_K = 4 \theta^2/\phi^2.
\end{displaymath}
If we assume $\lambda_1 / \lambda_0 = 0.2$ and $\theta = 3.261$ (see
Section 4.3),
\begin{displaymath}
  d_K = 4 \, (3.261)^2 / (\log 0.2)^2 = 16.42.
\end{displaymath}


\subsection{Repeated measures}

Many clinical trials are designed to measure subjects repeatedly over the
course of the trial, and define as the primary outcome the change or slope
over time.  For such trials, the difference between treatment groups can
be tested using the estimated slopes from each group using
\begin{displaymath}
  Z_k = (\bar B_T^{(k)} - \bar B_C^{(k)}) /  \sqrt{U_T^{(k)} + U_C^{(k)}}
\end{displaymath}
where $\bar B_T^{(k)}$ and $\bar B_C^{(k)}$ are the average of the slopes
estimated for patients in the treatment and control groups at the $k^{th}$
interim analysis, and $U_T^{(k)}$ and $U_C^{(k)}$ are their variances.
The sequentially computed $Z_k$ have been shown to have the required
Brownian motion structure when the variance parameters are known
(Reboussin, Lan \& DeMets, 1992; Wu \& Lan, 1992).  Lan, Reboussin \&
DeMets (1994) show
\begin{displaymath}
  {\rm E} (Z_K) =
       \frac{B_T - B_C}{\sigma_B} \sqrt{\frac{\hat I}{4}}
\end{displaymath}
where $B_T$ and $B_C$ are the mean population slopes, $\sigma^2_B$ is the
between patient variance of the slopes, and $\hat I$ is the natural
estimate of total information at the end of the trial.  For the comparison
of means and binomial proportions, $\hat I = 2 n_K$, but in this case, the
natural estimate of total information, denoted $\hat i$, is the sum of the
natural estimates of information for each patient:
\begin{displaymath}
  \hat i =
      \left[1+ R/\sum (t_j - \overline t)^2 \right]^{-1}
\end{displaymath}
where $R$ is the ratio of within to between patient variance.  For design
purposes, we may assume an identical number and timing of measurements for
all patients, so that $\hat I$ is $2 n_K \, \hat i$.  Then
\begin{displaymath}
  \delta_S = \frac{B_T - B_C}{\sqrt{2 \sigma_B^2}}
\end{displaymath}
and
\begin{displaymath}
  \theta = \frac{B_T - B_C}{\sqrt{2 \sigma_B^2}} \sqrt{n_K \hat i}
\end{displaymath}
so
\begin{eqnarray*}
  n_K = \theta^2 \frac{2 \sigma_B^2}{(B_T - B_C)^2} \, \frac{1}{\hat i}
      = \theta^2 \frac{2 \sigma_B^2}{(B_T - B_C)^2}
      \left[1+ R/\sum (t_j - \overline t)^2 \right].
\end{eqnarray*}
If a sufficient number of observations are taken on each patient, the term
$1+ R/\sum (t_j - \overline t)^2$ is nearly one (Lan, Reboussin
\& DeMets, 1994), so that the power computations are similar to the normal
case.


\section{Using the Program to Sequentially Analyze a Trial}
\label{sec-use}

We describe how to run the program using data from the Beta-Blocker Heart
Attack Trial or BHAT (Beta-Blocker Heart Attack Trial Research Group,
1982).  BHAT, a study sponsored by the National Heart, Lung and Blood
Institute, was designed to test whether long term use of propranolol by
patients with recent heart attack reduced mortality.  The following example
does not correspond exactly to what was actually done for BHAT, though it
is similar.  From June 1978 to October 1980, 3837 patients were randomized
to either propranolol (1916 patients) or placebo (1921 patients).
Follow-up was originally scheduled to end in June 1982.  The total
information D (number of deaths by June 1982) was never observed since the
trial was terminated early in October 1981.  The value of D was estimated
to be 628 when BHAT was designed, but with the data available in September
1982, was estimated to be around 400 (Lan \& DeMets, 1989).  In the six
Policy and Data Monitoring Board meetings (May 1979, October 1979, March
1980, October 1980, April 1981, and October 1981), the observed number of
deaths were (56, 77, 126, 177, 247, 318) and normalized log-rank
statistics were (1.68, 2.24, 2.37, 2.30, 2.34, 2.82).


\subsection{Computing boundaries with a spending function}

Let $t_c$ denote calendar time measured from the beginning of the trial,
and $T_c$ denote the maximum duration in calendar time. Let $\tau$ be the
information fraction or ``information time'', which must often be estimated
by $\hat{\tau}$, some function either of calendar time or number of
observed patients or events.  We begin with an example using only calendar
time.


\subsubsection*{Example with calendar time}

Set $t_c = 0$ in June 1978 and assume the maximum duration is $T_c = 48$
months, which corresponds to June 1982.  Then the calendar times for
interim analyses correspond to (11, 16, 21, 28, 34, 40) months after the
start of the trial.  We estimate $\tau$ as a function of calendar time by
$\hat{\tau} = t_c/T_c = t_c/48$, so the information times are (0.2292,
0.3333, 0.4375, 0.5833, 0.7083, 0.8333), and adopt the spending function
$\alpha^{\star}(\tau) = \alpha \tau$ to construct a data monitoring boundary.
This corresponds to $\alpha^{\star}_3(\tau)$ in Lan \& DeMets (1983) and Kim \&
DeMets (1987a).  The original BHAT design had a two-sided significance
level of 0.05.

When the data were monitored in May 1979, $t_{c1} = 11$,
$\hat{\tau_1}=11/48=0.2292$ and $\alpha^{\star}(\hat{\tau_1})=0.025 \times 0.2292
= 0.0057$.  The program produces a boundary value of $b_1 = 2.53$: if $Z_1$
is standard normal, $\Pr(Z_1 \geq 2.53) = 0.0057$.  In October 1979, $t_{c2}
= 16$, $\hat{\tau_2} = 16/48 = 0.3333$, and $\alpha^{\star}(\hat{\tau_2}) =
0.0083$.  Ignoring the observed number of deaths and using only calendar
time, the calculation proceeds as follows.  Suppose $Z_1$ and $Z_2$ are
standard normal with correlation coefficient
  \begin{math}
    \rho_{12}= (0.2292/0.3333)^{1/2}=0.8293.
  \end{math}
We wish to find $b_2$ such that
\begin{math}
  \Pr(Z_1 \geq 2.53 \, {\rm or} \, Z_2 \geq b_2) = 0.0083.
\end{math}
This solution requires some numerical integration which the program
performs.  In fact, this equality is satisfied if $b_2=2.61$.

{\singlespace
In this example, after specifying Option 1, the user is prompted for
  \begin{itemize}
    \item the number interim analyses (2),
    \item whether the analyses are equally spaced (no),
    \item times of the interim analyses (0.2292, 0.3333),
    \item whether a second time scale for information will be entered (Lan and
          DeMets, 1989) (no),
    \item the overall significance level (.05)
    \item whether the test was one-sided or two-sided symmetric (2),
    \item which $\alpha^{\star}$ function to apply
      ($\alpha^{\star}_3$)
    \item whether the boundary values should be truncated (no)
  \end{itemize}
(see Section 4.4).  The} program returns $b_1 = 2.53$ and $b_2 = 2.61$.
For the third analysis, we enter 0.2292, 0.3333, and 0.4375.  The program
returns $b_1$ and $b_2$ as before, plus $b_3 = 2.57$.  Continuing in this
manner, we obtain the boundary values (2.53, 2.61, 2.57, 2.47, 2.43, 2.38).


\subsubsection*{Example with information}

We now repeat the above calculation using the information in the
number of deaths.  Assuming the total information is the number of
expected events, $D = 628$, the information fractions are (56/628,
77/628, 126/628, 177/628, 247/628, 318/628), or (0.0892, 0.1226,
0.2006, 0.2818, 0.3933, 0.5064).  Then at the second interim analysis,
the program would ask for {\singlespace \vspace*{-1em} 
\begin{itemize}
\item the number interim analyses (2), 
\item whether the analyses are equally spaced (no), 
\item times of the interim analyses (0.0892, 0.1226),
\item whether a second time scale for information will be entered (Lan and
DeMets, 1989) (no), 
\item the overall significance level (.05) 
\item whether the test was one-sided or two-sided symmetric (2), 
\item which $\alpha^{\star}$ function to apply ($\alpha^{\star}_3$).  
\item whether the boundary values should be truncated (no)
\end{itemize} The}
information fractions are treated as times.  Since we do not enter the
information separately apart from the information fractions, the answer to
the question on a second time scale is ``no''.  The output boundary values
are (2.84, 2.97).  At the sixth analysis, when the additional times are
input, the resulting boundary values are (2.84, 2.97, 2.79, 2.72, 2.61,
2.54).

\subsubsection*{Two time scales}

Some users may be familiar with the use of both information and calendar
time as described in Lan \& DeMets (1989) and Lan, Reboussin \& DeMets
(1994).  The program includes such an option.  We will use the percent of
elapsed calendar time to determine how much type I error probability is to
be spent, but for the correlation of successive test statistics, we will
use the information in the number of deaths.  The first boundary is computed
exactly as above.  For the analysis in October 1979, at 16 months, $t_{c2}
= 16$, $\hat{\tau_2} = 16/48 = 0.3333$, and $\alpha^{\star}(\hat{\tau_2}) =
0.0083$ also just as before.  To evaluate $b_2$, note that even though
$\tau_2 = 77/D$ is unknown, $\tau_1/\tau_2 = 56/77 = 0.7273$ is observed.
If $Z_1$ and $Z_2$ are standard normal then the correlation coefficient
$\rho_{12} = \sqrt{0.7273} = 0.8528$, and the solution to \begin{math}
\Pr(Z_1 \geq 2.53 \, {\rm or} \, Z_2 \geq b_2) = 0.0083 \end{math} is $b_2 =
2.59$.  The program asks the same questions as before (see Section 4.6).
Since the times entered were based on the percent of elapsed calendar time,
it is desirable to use the information available in the number of deaths.
When the question on a second time scale for information is asked, we
answer ``yes'' and enter the information for each analysis, which is the
number of deaths in this example.  The resulting boundaries are (2.53, 2.59,
2.63, 2.50, 2.51, 2.47) for the six data monitoring points of BHAT, and
this boundary is crossed at $t_{c6} = 40$ or in October of 1981.  This is
the same as the result given for the example in Lan \& DeMets (1989).



\subsection{Computing confidence intervals}

Kim \& DeMets (1987b) detail the theory for confidence intervals
following early termination using group sequential tests.  Suppose that
a trial has been stopped at the $k^{th}$ analysis with boundary values
\begin{math}
  b_1,\,\ldots,\,b_k,
\end{math}
and with final standardized estimate of treatment difference $Z_k$.  The
confidence interval is based on computing upper exit probabilities
associated with
\begin{math}
  b_1,\,\ldots,\,b_{k-1},\,Z_k.
\end{math}

Continuing with the previous example, the final observed standardized
statistic was 2.82, and suppose that a 95 percent confidence interval is
desired.  The program prompts for {\singlespace
\vspace*{-1em}
  \begin{itemize}
    \item the number of analyses (6),
    \item whether the analyses are equally spaced between 0 and 1 (no),
    \item the information times of the analyses (.2292, .3333, etc.),
    \item whether a spending function will be used (no),
    \item whether the boundary is one or two sided (2),
    \item whether the two sided boundary is symmetric (yes),
    \item the boundaries to be evaluated (2.53, 2.61, etc.).
    \item the value of the standardized statistic at the last analysis
          (2.82),
    \item the confidence level (0.95).
  \end{itemize}
Computation} of confidence intervals is the most time consuming of the
four options since it involves a linear search.  The program outputs the
result (0.1881, 4.9347) (see Section 4.7).

Using the equation $\theta = \phi \sqrt{d/4}$ we can translate this
interval into an interval for $\phi$.  The statistic is based on 318
events, so $0.1881 = \phi \sqrt{318/4}$, or $\phi = 0.021$ is
the lower bound.  Repeating this computation for the upper bound, we obtain
(0.021, 0.553) as a 95\% confidence interval for $\phi$.


\section{Examples with Output}
\label{sec-examples}

This section contains examples of interactive sessions with the
program, which were used for the examples considered in Sections 2 and
3.

\subsection{Normally distributed data}

This program output related to the first example in Section 2.1.  For
this example, we use 5 equally spaced interim analyses (0.2, 0.4, 0.6,
0.8, and 1.0) with two--sided O'Brien--Fleming boundaries and $\alpha =
0.05$.  We first determine the boundaries and then for these boundaries,
determining the drift parameter $\theta$ to calculate a sample size.
{\singlespace {\scriptsize
\begin{verbatim}
           PROGRAM PROMPTS                                                USER INPUT
 Is this an interactive session? (1=yes,0=no)
                                                                              y
 interactive =            1
 Enter number for your option:
 (1) Compute bounds for given spending function.
 (2) Compute drift for given power and bounds
 (3) Compute probabilities for given bounds.
 (4) Compute confidence interval.
                                                                              1
 Option 1: You will be prompted for a spending function.
 Number of interim analyses?
                                                                              5
          5 interim analyses.
 Equally spaced times between 0 and 1? (1=yes,0=no)
                                                                              y
Analysis times:    0.200    0.400    0.600    0.800    1.000
 Do you wish to specify a second time/information scale? (e.g.
 number of patients or number of events, as in Lan & DeMets 89?)  (1=yes, 0=no)
                                                                              n
 Overall significance level? (>0 and <=1)
                                                                              .05
 alpha = 0.050
 One(1) or two(2)-sided symmetric?
                                                                              2
2.-sided test
 Use function? (1-5)
 (1) OBrien-Fleming type
 (2) Pocock type
 (3) alpha * t
 (4) alpha * t^1.5
 (5) alpha * t^2
                                                                              1
 Use function alpha-star           1
 Do you wish to truncate the standardized bounds? (1=yes, 0=no)               n

 Bounds will not be truncated.

This program generates two-sided symmetric boundaries.
n =  5
alpha = 0.050
use function for the lower boundary = 1
use function for the upper boundary = 1
       Time                Bounds         alpha(i)-alpha(i-1)   cum alpha
    0.20            -4.8769     4.8769         0.00000         0.00000
    0.40            -3.3569     3.3569         0.00079         0.00079
    0.60            -2.6803     2.6803         0.00683         0.00762
    0.80            -2.2898     2.2898         0.01681         0.02442
    1.00            -2.0310     2.0310         0.02558         0.05000

 Do you want to see a graph? (1=yes,0=no)
                                                                              y
\end{verbatim}
{\samepage
\begin{verbatim}
        :
    5.00:         *
    4.60:
    4.20:
    3.80:
    3.40:                  *
    3.00:
    2.60:                          *
    2.20:                                   *       *
    1.80:
    1.40:
    1.00:
    0.60:
    0.20:
   -0.20:
   -0.60:
   -1.00:
   -1.40:
   -1.80:
   -2.20:                                   *       *
   -2.60:                          *
   -3.00:
   -3.40:                  *
   -3.80:
   -4.20:
   -4.60:
   -5.00:         *
        ...............................................
          0  .1  .2   .3  .4  .5  .6  .7   .8  .9   1

 Done.
\end{verbatim}}}}

Once these initial boundaries are obtained, to compute the required sample
size, we must find the drift parameter corresponding to the desired power.
In the program, this is option 2.  We enter the times and boundary values
and select the desired power.  Alternatively, drift parameters for some
potential analysis scenarios are contained in Kim \& DeMets (1992).  In our
example, a drift parameter of 3.2788 gives a power of 0.90.  {\singlespace
{\scriptsize
\begin{verbatim}
           PROGRAM PROMPTS                                                USER INPUT
 Is this an interactive session? (1=yes,0=no)
                                                                              y
 interactive =            1
 Enter number for your option:
 (1) Compute bounds for given spending function.
 (2) Compute drift for given power and bounds
 (3) Compute probabilities for given bounds.
 (4) Compute confidence interval.
                                                                              2
 Option 2: You will be prompted for bounds and a power level.
 Number of interim analyses?
                                                                              5
          5 interim analyses.
 Equally spaced times between 0 and 1? (1=yes,0=no)
                                                                              y
Analysis times:    0.200    0.400    0.600    0.800    1.000
 Are you using a spending function to determine bounds? (1=yes,0=no)
                                                                              y
 Spending function will determine bounds.
 Overall significance level? (>0 and <=1)
                                                                              .05
 alpha = 0.050
 One(1) or two(2)-sided symmetric?
                                                                              2
2.-sided test
 Use function? (1-5)
 (1) OBrien-Fleming type
 (2) Pocock type
 (3) alpha * t
 (4) alpha * t^1.5
 (5) alpha * t^2
                                                                              1
 Use function alpha-star           1
 Do you wish to truncate the standardized bounds? (1=yes, 0=no)               n

 Bounds will not be truncated.
       Time                Bounds
    0.20            -4.8769     4.8769
    0.40            -3.3569     3.3569
    0.60            -2.6803     2.6803
    0.80            -2.2898     2.2898
    1.00            -2.0310     2.0310
 Desired power? (>0 and <=1)
                                                                              .9
 Power is 0.900

    n =  5, drift =  3.2788

look     time      lower    upper     exit  probability    cum exit pr

 1       0.20    -4.8769   4.8769        0.00032           0.00032
 2       0.40    -3.3569   3.3569        0.09939           0.09971
 3       0.60    -2.6803   2.6803        0.34658           0.44629
 4       0.80    -2.2898   2.2898        0.29966           0.74595
 5       1.00    -2.0310   2.0310        0.15405           0.90000

 Done.
\end{verbatim}}}
A drift of 3.28 was used in Section 2.1.1 to compute the required sample
size for 90\% power, which was 48.44 patients per arm.

Consider another sample size determination based on a different
initial analysis plan.  This set of analyses will be planned for
unequally spaced time points 0.1, 0.4, 0.75, 1.0, but other features
of the test are the same.  The program determines the corresponding
drift parameter. {\singlespace {\scriptsize
\begin{verbatim}
           PROGRAM PROMPTS                                                USER INPUT
 Is this an interactive session? (1=yes,0=no)
                                                                              y
 interactive =            1
 Enter number for your option:
 (1) Compute bounds for given spending function.
 (2) Compute drift for given power and bounds
 (3) Compute probabilities for given bounds.
 (4) Compute confidence interval.
                                                                              2
 Option 2: You will be prompted for bounds and a power level.
 Number of interim analyses?
                                                                              4
          4 interim analyses.
 Equally spaced times between 0 and 1? (1=yes,0=no)
                                                                              n
 Times of interim analyses: (>0 & <=1)
                                                                       .1 .4 .75 1.0
Analysis times:    0.100    0.400    0.750    1.000
 Are you using a spending function to determine bounds? (1=yes,0=no)
                                                                              y
 Spending function will determine bounds.
 Overall significance level? (>0 and <=1)
                                                                            .05
 alpha = 0.050
 One(1) or two(2)-sided symmetric?
                                                                              2
2.-sided test
 Use function? (1-5)
 (1) OBrien-Fleming type
 (2) Pocock type
 (3) alpha * t
 (4) alpha * t^1.5
 (5) alpha * t^2
                                                                              1
 Use function alpha-star           1
 Do you wish to truncate the standardized bounds? (1=yes, 0=no)               n

 Bounds will not be truncated.
       Time                Bounds
    0.10            -6.9914     6.9914
    0.40            -3.3569     3.3569
    0.75            -2.3449     2.3449
    1.00            -2.0125     2.0125
 Desired power? (>0 and <=1)
                                                                             .9
 Power is 0.900

    n =  4, drift =  3.2696

look     time      lower    upper     exit  probability    cum exit pr

 1       0.10    -6.9914   6.9914        0.00000           0.00000
 2       0.40    -3.3569   3.3569        0.09871           0.09871
 3       0.75    -2.3449   2.3449        0.58876           0.68746
 4       1.00    -2.0125   2.0125        0.21254           0.90000

 Done.
\end{verbatim}}}
The sample size is computed
\begin{displaymath}
  n_K = \left( \frac{\theta}{\delta_S} \right)^2 = 
        \left( \frac{3.27}{0.4714} \right)^2  = 48.11.
\end{displaymath}
Notice that the different timing of interim analyses has little impact on
the sample size needed to achieve 90\% power.


\subsection{Binomially distributed data}

In much the same manner as was done to compare two means from a normal
population, we can compare two proportions from a binomial population.
Recall the example from Section 2.2.1.  We use option 2 to determine the
drift parameter for a power of 90\% given one sided 0.05 Pocock boundaries
and five equally spaced analyses: {\singlespace {\scriptsize
\begin{verbatim}
           PROGRAM PROMPTS                                                USER INPUT
 Is this an interactive session? (1=yes,0=no)
                                                                              y
 interactive =            1
 Enter number for your option:
 (1) Compute bounds for given spending function.
 (2) Compute drift for given power and bounds
 (3) Compute probabilities for given bounds.
 (4) Compute confidence interval.
                                                                              2
 Option 2: You will be prompted for bounds and a power level.
 Number of interim analyses?
                                                                              5
          5 interim analyses.
 Equally spaced times between 0 and 1? (1=yes,0=no)
                                                                              y
Analysis times:    0.200    0.400    0.600    0.800    1.000
 Are you using a spending function to determine bounds? (1=yes,0=no)
                                                                              y
 Spending function will determine bounds.
 Overall significance level? (>0 and <=1)
                                                                              .05
 alpha = 0.050
 One(1) or two(2)-sided symmetric?
                                                                              1
1.-sided test
 Use function? (1-5)
 (1) OBrien-Fleming type
 (2) Pocock type
 (3) alpha * t
 (4) alpha * t^1.5
 (5) alpha * t^2
                                                                              2
 Use function alpha-star           2
 Do you wish to truncate the standardized bounds? (1=yes, 0=no)               n

 Bounds will not be truncated.
       Time                Bounds
    0.20            -8.0000     2.1762
    0.40            -8.0000     2.1437
    0.60            -8.0000     2.1132
    0.80            -8.0000     2.0895
    1.00            -8.0000     2.0709
 Desired power? (>0 and <=1)
                                                                              .9
 Power is 0.900

    n =  5, drift =  3.2055

look     time      lower    upper     exit  probability    cum exit pr

 1       0.20    -8.0000   2.1762        0.22884           0.22884
 2       0.40    -8.0000   2.1437        0.25845           0.48729
 3       0.60    -8.0000   2.1132        0.19989           0.68718
 4       0.80    -8.0000   2.0895        0.13238           0.81956
 5       1.00    -8.0000   2.0709        0.08044           0.90000

 Done.
\end{verbatim}}}


\subsubsection{The impact of changing frequency}

Even if the interim analyses actually performed during the study are not
equally spaced, the power is not greatly affected.  This can be seen in the
following example.  Recall our original plan had looks at 0.2, 0.4, 0.6,
0.8 and 1.0 and a target power of 90\%.  Suppose instead the looks occur at
0.2, 0.5, 0.6, 0.8, and 1.0.  Option 3 generates appropriate boundaries and
computes the power for a drift of 3.21.  As shown, the power is not
seriously affected.  {\singlespace {\scriptsize
\begin{verbatim} 
           PROGRAM PROMPTS                                                USER INPUT
 Is this an interactive session? (1=yes,0=no)
                                                                              y
 interactive =            1
 Enter number for your option:
 (1) Compute bounds for given spending function.
 (2) Compute drift for given power and bounds
 (3) Compute probabilities for given bounds.
 (4) Compute confidence interval.
                                                                              3
 Option 3: You will be prompted for bounds or a spending
 function to compute them.
 Number of interim analyses?
                                                                              5
          5 interim analyses.
 Equally spaced times between 0 and 1? (1=yes,0=no)
                                                                              n
 Times of interim analyses: (>0 & <=1)
                                                                     .2 .5 .6 .8 1.0
Analysis times:    0.200    0.500    0.600    0.800    1.000
 Are you using a spending function to determine bounds? (1=yes,0=no)
                                                                              y
 Spending function will determine bounds.
 Overall significance level? (>0 and <=1)
                                                                             .05
 alpha = 0.050
 One(1) or two(2)-sided symmetric?
                                                                              1
1.-sided test
 Use function? (1-5)
 (1) OBrien-Fleming type
 (2) Pocock type
 (3) alpha * t
 (4) alpha * t^1.5
 (5) alpha * t^2
                                                                              2
 Use function alpha-star           2
 Do you wish to truncate the standardized bounds? (1=yes, 0=no)               n

 Bounds will not be truncated.
       Time                Bounds
    0.20            -8.0000     2.1762
    0.50            -8.0000     2.0435
    0.60            -8.0000     2.1609
    0.80            -8.0000     2.0866
    1.00            -8.0000     2.0680
 Do you wish to use drift  parameters? (1=yes, 0=no)                          y
 How many drift parameters do you wish to enter?
                                                                              1
          1 drift parameters.
 Enter drift parameters:
                                                                            3.21
Drift parameters:    3.210
 Drift is equal to the standard treatment difference times the square
 root of total information per arm.

    n =  5, drift =  3.2100

look     time      lower    upper     exit  probability    cum exit pr

 1       0.20    -8.0000   2.1762        0.22945           0.22945
 2       0.50    -8.0000   2.0435        0.38289           0.61234
 3       0.60    -8.0000   2.1609        0.07757           0.68991
 4       0.80    -8.0000   2.0866        0.13220           0.82211
 5       1.00    -8.0000   2.0680        0.07941           0.90152

 Done.
\end{verbatim}}}



\subsection{Survival data}

Referring to the previous survival example in Section 2.3, assume that
three equally spaced analyses were initially planned for this study, and
that test was to have 90\% power.  The following output from the program
illustrates the Brownian motion drift parameter of 3.261 will give the
desired power.  {\singlespace {\scriptsize
\begin{verbatim}
        PROGRAM PROMPTS                                                USER INPUT

    Is this an interactive session? (1=yes,0=no)
                                                                           y
 interactive =            1
 Enter number for your option:
 (1) Compute bounds for given spending function.
 (2) Compute drift for given power and bounds
 (3) Compute probabilities for given bounds.
 (4) Compute confidence interval.
                                                                           2
 Option 2: You will be prompted for bounds and a power level.
 Number of interim analyses?
                                                                           3
          3 interim analyses.
 Equally spaced times between 0 and 1? (1=yes,0=no)
                                                                           y
Analysis times:    0.333    0.667    1.000
 Are you using a spending function to determine bounds? (1=yes,0=no)
                                                                           y
 Spending function will determine bounds.
 Overall significance level? (>0 and <=1)
                                                                          .05
 alpha = 0.050
 One(1) or two(2)-sided symmetric?
                                                                           2
2.-sided test
 Use function? (1-5)
 (1) OBrien-Fleming type
 (2) Pocock type
 (3) alpha * t
 (4) alpha * t^1.5
 (5) alpha * t^2
                                                                           1
 Use function alpha-star           1
 Do you wish to truncate the standardized bounds? (1=yes, 0=no)               n

 Bounds will not be truncated.
       Time                Bounds
    0.33            -3.7103     3.7103
    0.67            -2.5114     2.5114
    1.00            -1.9930     1.9930
 Desired power? (>0 and <=1)
                                                                          .90
 Power is 0.900

    n =  3, drift =  3.2608

look     time      lower    upper     exit  probability    cum exit pr

 1       0.33    -3.7103   3.7103        0.03380           0.03380
 2       0.67    -2.5114   2.5114        0.52651           0.56031
 3       1.00    -1.9930   1.9930        0.33969           0.90000

 Done.
\end{verbatim}}}



\subsection{Computing bounds during analysis of a trial}

This is an interactive session using the BHAT data and calendar time as the
only time scale.  The input sequence is described in Section~\ref{sec-use}.
{\singlespace {\scriptsize 
\begin{verbatim}
           PROGRAM PROMPTS                                                USER INPUT

 Is this an interactive session? (1=yes,0=no)
                                                                              y
 interactive =            1
 Enter number for your option:
 (1) Compute bounds for given spending function.
 (2) Compute drift for given power and bounds
 (3) Compute probabilities for given bounds.
 (4) Compute confidence interval.
                                                                              1
 Option 1: You will be prompted for a spending function.
 Number of interim analyses?
                                                                              2
          2 interim analyses.
 Equally spaced times between 0 and 1? (1=yes,0=no)
                                                                              n
 Times of interim analyses: (>0 & <=1)
                                                                        .2292 .3333
Analysis times:    0.229    0.333
 Do you wish to specify a second time/information scale? (e.g.
 number of patients or number of events, as in Lan & DeMets 89?)  (1=yes, 0=no)
                                                                             no
 Overall significance level? (>0 and <=1)
                                                                            .05
 alpha = 0.050
 One(1) or two(2)-sided symmetric?
                                                                             2
2.-sided test
 Use function? (1-5)
 (1) OBrien-Fleming type
 (2) Pocock type
 (3) alpha * t
 (4) alpha * t^1.5
 (5) alpha * t^2
                                                                             3
 Use function alpha-star           3
 Do you wish to truncate the standardized bounds? (1=yes, 0=no)               n

 Bounds will not be truncated.

This program generates two-sided symmetric boundaries.
n =  2
alpha = 0.050
use function for the lower boundary = 3
use function for the upper boundary = 3
       Time                Bounds         alpha(i)-alpha(i-1)   cum alpha
    0.23            -2.5284     2.5284         0.01146         0.01146
    0.33            -2.6098     2.6098         0.00520         0.01667

 Do you want to see a graph? (1=yes,0=no)
                                                                             n

 Done.
\end{verbatim}}} 

In this case, the program outputs the number of analyses so far, the type I
error specified, the use function chosen, the times, the computed
boundaries, and the type I error ``spent'' at each analysis so far.


\subsection{Using the program noninteractively}

Some users may want to use the program noninteractively.  This can be done
by preparing an input file with the appropriate format.  Each question is
answered on its own line in the input file, and the answer to the first
question must be ``no'' or ``0''.  Here is an input file which reproduces
the above interactive session: {\singlespace {\scriptsize
\begin{verbatim}
0                                     # noninteractive
1                                     # option 1: bounds
2                                     # number of analyses
0                                     # equally spaced? (0=no)
.2292 .3333                           # times of analyses
0                                     # second time scale? (0=no)
.05                                   # alpha
2                                     # 1 or 2 sided test
3                                     # use function (1-5)
0                                     # truncate boudaries (0=no)
0                                     # show graph? (0=no)
0                                     # start again? (0=no)
\end{verbatim}}
The resulting output is {\scriptsize \singlespace
\begin{verbatim}
 Is this an interactive session? (1=yes,0=no)
 interactive =            0
          2 interim analyses.
Analysis times:    0.229    0.333
 alpha = 0.050
2.-sided test
 Use function alpha-star           3


This program generates two-sided symmetric boundaries.
n =  2
alpha = 0.050
use function for the lower boundary = 3
use function for the upper boundary = 3
       Time                Bounds         alpha(i)-alpha(i-1)   cum alpha
    0.23            -2.5284     2.5284         0.01146         0.01146
    0.33            -2.6098     2.6098         0.00520         0.01667

 Do you want to see a graph? (1=yes,0=no)

 Done.
\end{verbatim}}}


\subsection{Using information to compute boundaries during analysis}

For this session, the numbers of events were entered as information, as
described in Section 3.1.
{\singlespace {\scriptsize 
\begin{verbatim}
           PROGRAM PROMPTS                                                USER INPUT

 Is this an interactive session? (1=yes,0=no)
                                                                              y
 interactive =            1
 Enter number for your option:
 (1) Compute bounds for given spending function.
 (2) Compute drift for given power and bounds
 (3) Compute probabilities for given bounds.
 (4) Compute confidence interval.
                                                                              1
 Option 1: You will be prompted for a spending function.
 Number of interim analyses?
                                                                              6
          6 interim analyses.
 Equally spaced times between 0 and 1? (1=yes,0=no)
                                                                              n
 Times of interim analyses: (>0 & <=1)
                                                 .2292 .3333 .4375 .5833 .7083 .8333
Analysis times:    0.229    0.333    0.438    0.583    0.708    0.833
 Do you wish to specify a second time/information scale? (e.g.
 number of patients or number of events, as in Lan & DeMets 89?)  (1=yes, 0=no)
                                                                              y
 Second scale will estimate covariances.
 Information:
                                                               56 77 126 177 247 318
Information  56.000  77.000 126.000 177.000 247.000 318.000
 Overall significance level? (>0 and <=1)
                                                                             .05
 alpha = 0.050
 One(1) or two(2)-sided symmetric?
                                                                              2
2.-sided test
 Use function? (1-5)
 (1) OBrien-Fleming type
 (2) Pocock type
 (3) alpha * t
 (4) alpha * t^1.5
 (5) alpha * t^2
                                                                              3
 Use function alpha-star           3
 Do you wish to truncate the standardized bounds? (1=yes, 0=no)               n

 Bounds will not be truncated.

This program generates two-sided symmetric boundaries.
n =  6
alpha = 0.050
use function for the lower boundary = 3
use function for the upper boundary = 3
    Time    Information     Bounds      alpha(i)-alpha(i-1)  cum alpha
  0.23       56.00    -2.5284   2.5284       0.01146       0.01146
  0.33       77.00    -2.5905   2.5905       0.00520       0.01667
  0.44      126.00    -2.6327   2.6327       0.00521       0.02187
  0.58      177.00    -2.5036   2.5036       0.00729       0.02916
  0.71      247.00    -2.5073   2.5073       0.00625       0.03542
  0.83      318.00    -2.4655   2.4655       0.00625       0.04166

 Do you want to see a graph? (1=yes,0=no)
                                                                             n

 Done.
\end{verbatim}} 
In} addition to the output described previously, the information is also
reported.



\subsection{Computing a confidence interval at the end of a trial}

In addition to the information needed to compute probabilities associated
with a set of boundaries, computing a confidence interval also requires the
last value of the standardized test statistic. {\singlespace {\scriptsize
\begin{verbatim}
           PROGRAM PROMPTS                                                USER INPUT

 Is this an interactive session? (1=yes,0=no)
                                                                              y
 interactive =            1
 Enter number for your option:
 (1) Compute bounds for given spending function.
 (2) Compute drift for given power and bounds
 (3) Compute probabilities for given bounds.
 (4) Compute confidence interval.
                                                                              4
 Option 4: You will be prompted for bounds and a confidence level.
 Number of interim analyses?
                                                                              6
          6 interim analyses.
 Equally spaced times between 0 and 1? (1=yes,0=no)
                                                                              n
 Times of interim analyses: (>0 & <=1)
                                                            .2292 .3333 .4375 .5833 .7083 .8333
Analysis times:    0.229    0.333    0.438    0.583    0.708    0.833
 Are you using a spending function to determine bounds? (1=yes,0=no)
                                                                             no
 You must enter a set of bounds.
 One(1)- or two(2)-sided?
                                                                              2
            2-sided test
 Symmetric bounds? (1=yes,0=no)
                                                                              y
 Two sided symmetric bounds.
 Enter upper bounds (standardized):
                                                                2.53 2.61 2.57 2.47 2.43 2.38
 Bounds entered.
       Time                Bounds
    0.23            -2.5300     2.5300
    0.33            -2.6100     2.6100
    0.44            -2.5700     2.5700
    0.58            -2.4700     2.4700
    0.71            -2.4300     2.4300
    0.83            -2.3800     2.3800
 Enter the standardized statistic at the last analysis:
                                                                            2.82
   Last value:                 2.8200
 Enter confidence level (>0 and <1):
                                                                             .95
 95. percent confidence interval

 Starting computation for lower limit . . .
 Lower limit computed, starting on upper limit . . .


 95. percent confidence interval: ( 0.1881, 4.9347)
 Drift is equal to the standard treatment difference times the square
 root of total information per arm.

 Done.
\end{verbatim}}
Translation} of the standardized parameter back to an estimate of the
difference between treatment groups is done in Section 3.2

\begin{center}
  {\bf Acknowledgements}
\end{center}

The authors wish to acknowledge of Kris Erlandson and Bill Ladd for
assistance in constructing examples, and Wen Wei for assistance in
programming.

\begin{thebibliography}{xx}

\item
Armitage, P., McPherson, C.~K. \& Rowe, B.~C.  (1969), `Repeated significance
  tests on accumulating data', {\em Journal of the Royal Statistical Society,
  Series {A}} {\bf 132},~235--244.

\item
{Beta-Blocker Heart Attack Trial Research Group}  (1982), `A randomized trial
  of propranolol in patients with acute myocardial infarction. {I, Mortality
  results.}', {\em Journal of the American Medical Association} {\bf
  246},~1707--1714.

\item
DeMets, D.~L., \& Lan, K.~K.~G. (1984), `An overview of sequential methods
and their applications in clinical trials', {\em Communications in
Statistics, Theory and Methods}, {\bf 13},~2315--2338.

\item
Hwang, I.~K., Shih, W.~J. \& deCani, J.~S. (1990), `Group sequential
designs using a family of type I error probability spending functions',
{\em Statistics in Medicine}, {\bf 9},~1439--1445.

\item
Kim, K. \& DeMets, D.~L.  (1987{\em a}), `Design and analysis of group
  sequential tests based on the type I error spending rate function', {\em
  Biometrika} {\bf 74},~149--154.

\item
Kim, K. \& DeMets, D.~L.  (1987{\em b}), `Confidence intervals following group
  sequential tests in clinical trials', {\em Biometrics} {\bf 43},~857--864.

\item
Kim, K. \& DeMets, D.~L.  (1992), `Sample size determination for group
  sequential clinical trials with immediate response', {\em Statistics in
  Medicine} {\bf 11},~1391--1399.

\item
Lan, K. K.~G. \& DeMets, D.~L.  (1983), `Discrete sequential boundaries for
  clinical trials', {\em Biometrika} {\bf 70},~659--663.

\item
Lan, K. K.~G. \& DeMets, D.~L.  (1989), `Group sequential procedures: calendar
  versus information time', {\em Statistics in Medicine} {\bf 8},~1191--1198.

\item
Lan, K. K. G. and Zucker, D. M., (1993) `Sequential monitoring of clinical
trials: the role of information and Brownian motion', {\em Statistics in
Medicine}  {\bf 12},~753--765.

\item
Lan, K. K.~G., Reboussin, D.~M. \& DeMets, D.~L.  (1994), `Information and
  information fractions for design and sequential monitoring of clinical
  trials', {\em Communications in Statistics, Part {A}---Theory and Methods}
  {\bf 23},~403--420.

\item
McPherson, C.~K. \& Armitage, P.  (1971), `Repeated significance tests on
  accumulating data when the null hypothesis is not true', {\em Journal of the
  Royal Statistical Society, Series {A}} {\bf 134},~15--25.

\item
O'Brien, P.~C. \& Fleming, T.~R.  (1979), `A multiple testing procedure for
  clinical trials', {\em Biometrics} {\bf 35},~549--556.

\item
Pocock, S.~J.  (1977), `Group sequential methods in the design and analysis of
  clinical trials', {\em Biometrika} {\bf 64},~191--199.

\item
{Reboussin92a}
Reboussin, D.~M., DeMets, D.~L., Kim, K. \& Lan, K. K.~G. (1992), Programs for
  computing group sequential boundaries using the {Lan-DeMets} method,
  Technical Report~60, Department of Biostatistics,
  University of Wisconsin-Madison.

\item
Reboussin, D. M., Lan, K. K. G. \& DeMets, D. L. (1992).
  Group sequential testing of longitudinal data.
  Technical Report~72, Department of Biostatistics,
  University of Wisconsin-Madison.

\item
  Wu, M.~C. \& Lan, K.~K.~G., (1992), `Sequential monitoring for comparison
  of changes in a response variable in clinical studies', {\em Biometrics}
  {\bf 48},~765--779.

\end{thebibliography}


\section*{Appendix}

\noindent
{\em Theory related to the computations.}

Consider a Brownian motion process in continuous time, $W(t)$, $t \in [0,1]$,
having unknown drift parameter $\theta$, which may be inspected at times
${t_i, \, i=1,\ldots,k}$.  We wish to test the hypothesis $H_0:
\theta \leq 0$ at each inspection time $t_i$ and proceed only if the test
fails to reject; that is, if $W(t_i)$ does not exceed some value, so that
the sequential test rejects if $Z(t_i) = W(t_i)/\sqrt{t_i} > b_i$.
Consider a sequence of boundaries, $b_1,\,\ldots,\,b_k$ applied at
times $t_1,\,\ldots,\,t_k$.  Let $g$ denote the standard normal density
function,
\begin{math}
 g(u) =
 \left({2\pi}\right)^{-1/2}
 \exp \left( - u^2/2 \right).
\end{math}
The probability distribution for $W$ at analysis $i$ is determined
recursively by
\begin{math}
 f_1(x) = g(x)
\end{math}
and
\begin{displaymath}
 f_i(x) =
 \int_{-\infty}^{b_{i-1}}
  g \left( (x-u)/\sigma_i \right) \,
  \sigma_i^{-1} \, f_{i-1}(u) \, du
\end{displaymath}
where $\sigma^2_i$ is the variance of $[W(t_i)-W(t_{i-1})]$, that is,
$\sigma^2_i = t_i - t_{i-1}$.  Integrating $f_i$ from $- \infty$ to $+
\infty$ gives the probability that the trial continues past the
$(i-1)^{th}$ analysis.

Computations at the first analysis involve only the standard normal density
and distribution function, but for the second and beyond, numerical
integration is necessary.  By applying Fubini's theorem, we have the
continuation probability at analysis $i$
\begin{eqnarray*}
P_i &=&
  \int_{-\infty}^{b_i} \int_{-\infty}^{b_{i-1}}
  g \left( (x-u)/\sigma_i \right) \,
  \sigma_i^{-1} \, f_{i-1}(u) du \, dx \nonumber \\
  &=& \int_{-\infty}^{b_{i-1}} \int_{-\infty}^{b_i}
  g \left( (x-u)/\sigma_i \right) \,
  \sigma_i^{-1} \, f_{i-1}(u) dx \, du \nonumber \\
  &=& \int_{-\infty}^{b_{i-1}}
  \Phi \left( (b_i-u)/\sigma_i \right) \,
  f_{i-1}(u) du.
\end{eqnarray*}
Note that only a single numerical integration is now required.  This
manipulation allows the use of simple, accurate approximations to the
normal distribution function to be used for computing $P_i$.  Extension of
the above to two sided tests is straightforward: if $a_i$ is the lower
bound, it can be substituted for $-\infty$ in the above integrals.

\vspace*{2em}
\noindent
{\em Description of computations.}

For the first analysis, which uses only the cumulative normal distribution,
we have $W(\tau_1) \, \sim \, N(\theta \tau_1, \tau_1)$.  The probability
calculated for exceeding the first upper boundary is
\begin{eqnarray*}
  P \left(\frac{W(\tau_1)}{\sqrt{\tau_1}} \, > \, b_1 \right)
  & = & P \left(W(\tau_1) \, > \, b_1 \sqrt{\tau_1} \right) \\
  & = & P \left(\frac{    W(\tau_1)             - \theta \tau_1}
                     {\sqrt{\tau_1}}
                     \, > \, 
                \frac{b_1 \sqrt{\tau_1} - \theta \tau_1}
                     {\sqrt{\tau_1}} \right) \\
  & = & P \left(Z > b_1 - \theta \sqrt{\tau_1} \right) \\
  & = & 1 - \Phi \left( b_1 - \theta \sqrt{\tau_1} \right).
\end{eqnarray*}
In the programs, given $(a_i, b_i), \, i > 1$, separate subroutines are
called to compute the exit probability, denoted $P_i$ and, if there are
more analyses to come, to compute $f_i$.  For the routine computing $P_i$,
a grid of values of $f_{i-1}(u)$ for $u \in (a_{i-1},b_{i-1})$, saved from
the previous step, is needed.  The grid size is standardized, so that it is
finer when the increment has a smaller standard deviation.  At each grid
point $u$, the quantity
\begin{math}
  [\Phi((b_i-u)/\sigma_i) - \Phi((a_i-u)/\sigma_i)] \, f_{i-1}(u)
\end{math}
is computed and stored in an array.  This array is then passed to a
numerical integration routine along with $a_{i-1},\,b_{i-1}$ and the grid
size, and $1-P_i$ is returned.  The other
subroutine computes $f_i$ for a grid of values between $a_i$ and $b_i$.
For each grid point, the grid of values of $f_{i-1}$ is needed.  Letting
$u$ denote a point in the grid from $a_{i-1}$ to $b_{i-1},$ and $x$ denote
a point in the grid from $a_i$ to $b_i$, the quantity
\begin{math}
  f_{i-1}(u) \: g((u-x)/\sigma_i)/\sigma_i
\end{math}
is computed and stored in an array.  As before, this array is passed to a
numerical integration routine, along with $a_{i-1},\,b_{i-1},$ and the grid
size, and $f_i(x)$ is obtained and stored for the next step.  Currently,
the numerical integration routine is a composite trapezoidal rule, which
appears to produce fairly accurate results.  Reboussin, DeMets, Kim \& Lan
(1992) present testing of the programs for computational accuracy and
simulations results for validity.  Their appendices contain listings of the
code.

\vspace*{2em}
\noindent
{\em Programming for spending functions.}

Boundaries and information fractions are related by the type I error
spending function.  The program contains five choices for these functions
in a single subroutine called {\tt alphas}.  The critical source code is:
{\singlespace \begin{verbatim}
c     Calculate probabilities according to use function.
      do 50 i=1,nn
         if (iuse .eq. 1) then
            pe(i)=2.d0*
     .       (1.d0-pnorm(znorm(1.d0-(alpha/side)/2.d0)/dsqrt(t(i))))
         else if (iuse .eq. 2) then
            pe(i)=(alpha/side)*dlog(1.d0 + (e-1.d0)*t(i))
         else if (iuse .eq. 3) then
            pe(i)=(alpha/side)*t(i)
         else if (iuse .eq. 4) then
            pe(i)=(alpha/side)*(t(i) ** 1.5d0)
         else if (iuse .eq. 5) then
            pe(i)=(alpha/side)*(t(i) ** 2.0d0)
c     Add other spending function options here: e.g.
c        else if (iuse.eq.6) then . . .
         else
            write(6,*) ' Warning: invalid use function.'
         end if
\end{verbatim}}

Additional spending functions can be added as ``silent'' options by editing
this section of code.  For example, here is the code for a spending
function which does not allow stopping until the trial is half over.  Once
half the information has accumulated, the type I error is spent uniformly
until the end of the trial.
{\singlespace \begin{verbatim}
         else if (iuse .eq. 6) then
            if (t(i).le.0.0) then
               pe(i)=0.0d0
            else
               pe(i)=(alpha/side)*(t(i) * 2.0d0 - 1.d0)
            end if
\end{verbatim}}
This could also be added to the input routine with some additional
programming effort.

\end{document}
